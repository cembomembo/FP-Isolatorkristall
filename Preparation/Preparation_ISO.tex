\documentclass[a4paper]{article}

% --- Page layout and spacing ---
\usepackage[top=3cm, left=3.5cm, right=3.5cm, bottom=3cm]{geometry}
\usepackage[utf8]{inputenc}      % input encoding
\usepackage[T1]{fontenc}         % font encoding
\usepackage[english]{babel}
\usepackage{setspace}
\setlength{\parindent}{0pt}      % paragraph indentation
\setlength{\parskip}{0.8em}      % space between paragraphs
\setstretch{1.2}                 % line spacing
\usepackage{tocloft}             % section spacing in ToC
\setlength{\cftbeforesecskip}{10pt}
\setlength{\cftbeforesubsecskip}{4pt}
\usepackage{titlesec}            % section title spacing
\titlespacing*{\section}{0pt}{5.0ex plus 1ex minus .2ex}{1.0ex plus .2ex}
\titlespacing*{\subsection}{0pt}{3.0ex plus .5ex minus .2ex}{0.8ex plus .2ex}
\titlespacing*{\subsubsection}{0pt}{2.0ex plus .5ex minus .2ex}{0.8ex plus .2ex}

% --- Math and symbols ---
\usepackage{amsmath, amssymb}    % standard math
\usepackage{empheq}              % boxed equations etc.
\DeclareMathOperator{\artanh}{artanh}
\DeclareMathOperator{\sgn}{sgn}
\usepackage{bm}                  % bold math symbols
\usepackage{cancel}              % strikeout in math
\usepackage{siunitx}             % proper units
\renewcommand{\arraystretch}{0.7}

% --- Graphics and floats ---
\usepackage{graphicx}
\usepackage{float}
\usepackage{wrapfig}
\usepackage[justification=centering]{caption}
\usepackage{subcaption}
\captionsetup[figure]{font=small}

% --- Layout helpers ---
\usepackage{boxedminipage}
\usepackage{enumitem}
\usepackage{afterpage}
\usepackage{changepage}
\usepackage{pdfpages}           % include external PDFs
\usepackage{esvect}             % nice vector arrows
\usepackage{hyperref}           % hyperlinks

% --- Bibliography setup ---
\usepackage{csquotes}
\usepackage[backend=biber,style=numeric,sorting=none]{biblatex}
\addbibresource{references.bib}

% --- Fonts ---
\usepackage{lmodern}            % Computer Modern look across TeX distros



% --- Title page ---
\title{\textbf{Optical Measurements on a Ruby Crystal}}
\author{
  \\Preparation Report \\\\\\\\\\\\
  \textbf{Cem Boyaci} \\
  cemb93@zedat.fu-berlin.de \\\\\\
  \textbf{Javier Bellido Roland}\\
  bellidoroj98@zedat.fu-berlin.de \\\\\\
  \textbf{Leon Goldammer} \\
  lg4278fu@zedat.fu-berlin.de \\\\
}
\date{}

\begin{document}
\maketitle
\thispagestyle{empty}

\section*{}
\begin{center}
\vspace{3cm}
Tutor: Christian Teutloff \\[1cm]
\textbf{Fortgeschrittenenpraktikum, WS 2025/2026}\\
Berlin, 17.11.2025\\
Freie Universität Berlin\\
Fachbereich Physik
\end{center}



% --- Table of contents ---
\clearpage
\renewcommand*\contentsname{\huge Contents}
{
  \pagenumbering{gobble}
  \tableofcontents
  \clearpage
}
\pagenumbering{arabic}



% --- Introduction ---
\newpage
\setcounter{page}{1} 
\section{Introduction}

Ruby is a red variety of corundum (Al$_2$O$_3$) whose colour originates from small amounts of Cr$^{3+}$ ions replacing Al$^{3+}$ sites in the crystal.
Unlike pure corundum, which is colourless and transparent, the chromium impurities introduce electronic states that interact with visible light and give rise to characteristic absorption features.
In this experiment we measure the wavelength-dependent transmission of a ruby crystal using a two-beam absorption spectrometer.
From the recorded absorption bands we aim to understand the optical origin of the ruby colour and relate the observed transitions to the crystal-field splitting of the Cr$^{3+}$ ions.
The experiment therefore provides a direct link between optical spectroscopy and the basic concepts of crystal-field theory as applied to doped insulating crystals.



% --- Physical Principles ---
\section{Physical Principles}
\subsection{Crystal Structure of Corundum}

Corundum (Al$_2$O$_3$) is a transparent, colorless insulating crystal in which each Al$^{3+}$ ion is surrounded by six O$^{2-}$ ions in a slightly distorted octahedral arrangement (Fig. \Ref{fig:corundum}).
The negatively charged oxygen ions create an electrostatic field around every cation site.
In the case of Al$^{3+}$, this field has almost no optical consequence, since aluminium has no electrons in its outer \(d\)-shell and therefore no low-energy electronic states that could be shifted or excited by the surrounding oxygen ions.
The Al$^{3+}$ sites behave essentially like stable point charges that hold the lattice together without interacting with visible light.
For this reason pure corundum is colorless. The situation changes, however, when an Al$^{3+}$ ion is replaced by another ion that does carry \(d\)-electrons, creating a substitutional point defect and introducing new electronic states into this same electrostatic environment.

\vspace{1em}
\begin{figure}[H]
    \centering
    \includegraphics[width=0.78\textwidth]{../resources/figures/CorundumStructure.jpg}
    \caption{The distorted-octahedral oxygen-ligand environment around an Al$^{3+}$ ion in corundum Al$_2$O$_3$ is shown in two ways, (A) and (B), and compared with a regular octahedron (C) \cite{webexhibits_causes_of_color}.}
    \label{fig:corundum}
\end{figure}


\subsection{Substitution of Al\textsuperscript{3+} by Cr\textsuperscript{3+}}

When an Al$^{3+}$ ion in corundum is replaced by a Cr$^{3+}$ ion, the overall charge of the lattice does not change because both ions carry the same charge.
The important difference is that Cr$^{3+}$ has three electrons in its outer $3d$ shell, while Al$^{3+}$ has none.
These $d$-electrons occupy orbitals that have specific spatial shapes and point in certain directions around the ion.
Because the Cr$^{3+}$ ion sits in the same octahedral environment as Al$^{3+}$, some of these $d$-orbitals point more directly toward the surrounding O$^{2-}$ ions than others.
The negatively charged oxygen ions therefore push on the $d$-electrons with different strengths depending on their orientation.
This causes the $d$-orbitals of the Cr$^{3+}$ ion to shift to different energies instead of all having the same energy as they would in a free ion.
As a consequence, the Cr$^{3+}$ ion acquires distinct electronic states that can absorb light of certain colours, which is why the presence of chromium changes the colour of pure corundum and turns it into ruby.


\subsection{xxx}

xxx


\subsection{xxx}

xxx



% --- Experimental Setup ---
\section{Experimental Setup}
\subsection{xxx}

some notes of what should be included in this section

 double beam absorption spectrometer
 - optical path and components
   - light source
   - beam splitting
   - intensity equalization
   - modulation
   - spectral dispersion
 - lock-in detection and signal extraction
   - noise rejection Principles
   - lock-in amplifier (understood)
   - differential Measurement
   - output signal 
   - input safety 
 - data aquisition
   - interfaces
   - data reading
   - software and storage
 - wavelength calibration setup
   - without chopper or lock-in amplifier
   - calibration source: mercury lamp
   - signal path and constraint
   - noise filtering

\subsection{xxx}

xxx



% --- Procedure ---
\section{Procedure}
\subsection{xxx}

xxx


\subsection{xxx}

xxx



% --- References ---
\newpage
\setstretch{1.0}
\printbibliography[heading=bibintoc]



\end{document}
